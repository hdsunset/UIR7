\chapter*{Реферат}
\thispagestyle{plain}

Пояснительная записка содержит 43 страницы.   Количество использованных источников~--15. Количество приложений ~-- 1.

Ключевые слова: шаблон, генерация файлов, Word-документ, LaTex, PDF-документ, .NET,  веб-сервис, веб-интерфейс.

    Целью данной учебно-исследовательской работы является исследование и разработка прототипа системы генерации документов на основе шаблонов.
    
    В первом разделе описывается изучение и сравнительный анализ формальных средств описания шаблонов объектов. Изучение правил подстановки при формировании конечных файлов.
             
Во втором разделе описан процесс разработки модели объектов данных и метаданных для описания шаблонов документов и механизмов связывания формальных параметров с фактическими информационными объектами. Также описана разработка концептуальной модели тестовой предметной области – генерации протоколов для ВКР.

    В третьем разделе описывается инженерная часть научно-исследовательской работы, а именно проектирование архитектуры разрабатываемой системы и разработка функциональной структуры информационной системы.
    
    В четвёртом разделе описывается программная реализация механизмов генерации Word-файлов с простейшим пользовательским интерфейсом, а также приводится пример создания тестового документа протокола ВКР.

