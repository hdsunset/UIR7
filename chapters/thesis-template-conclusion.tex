\chapter*{Заключение}
\addcontentsline{toc}{chapter}{Заключение}

В данной работе был рассмотрен процесс генерации документов Word программным путем по шаблону. Были изучены аспекты различных технологий, используемых при автоматической генерации содержимого документов, были разобраны основы работы с объектами Microsoft Office, в частности Word и Excel, а также был разобран функционал пользовательских пакетов С#, предназначенных для решения описанной задачи. Была построена концептуальная модель преметной области а также спроектирована архитектура разрабатываемой системы. В качестве тестового примера был сгенерирован простейший протокол ВКР, заполненный по шаблону данными, полученными из базы данных.
В дальнейшем планируется продолжит рассмотрение механизмов генерации документов Word и PDF программным путем, а также при реализации веб-сервиса. Для построения полноценной системы основным инструментом разработки веб-приложений была выбрана технология ASP.NET Web API 2.0, что позволит реализовать весь необходимый функционал веб-сервиса.
    В конечном итоге планируется реализовать веб-ориентированную систему генерации документов на основе шаблонов, которая может применяться для автоматического создания необходимых документов на основе имеющегося массива данных.
