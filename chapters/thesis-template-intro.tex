\chapter*{Введение}
\label{sec:afterwords}
\addcontentsline{toc}{chapter}{Введение}

В настоящее время непрерывно растёт объём документации работ на различных видов предприятиях, офисах и т.д. В подобных сферах, будь то разработка корпоративных приложений, бухгалтерия или образовательная система, очень часто приходится решать задачу выгрузки данных в документы — от небольших справок до больших отчетов. Следовательно становится всё более актуальным вопрос о создании решения для генерации docx/pdf документов, которое позволяет заполнять документы по шаблону, оформление которого можно менять в соответствующей среде без переписывания кода.

В первом разделе описывается изучение и сравнительный анализ формальных средств описания шаблонов объектов. Изучение правил подстановки при формировании конечных файлов.         

Во втором разделе описан процесс разработки модели объектов данных и метаданных для описания шаблонов документов и механизмов связывания формальных параметров с фактическими информационными объектами. Также описана разработка концептуальной модели тестовой предметной области – генерации протоколов для ВКР.

    В третьем разделе описывается инженерная часть научно-исследовательской работы, а именно проектирование архитектуры системы и разработка функциональной структуры информационной системы.
    
    В четвёртом разделе описывается программная реализация механизмов генерации Word-файлов, приводится пример генерации тестового докум	ента.



