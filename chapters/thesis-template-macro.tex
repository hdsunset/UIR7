\setlength\paperheight{297mm}
\setlength\paperwidth{210mm}

\usepackage[T2A]{fontenc}

\usepackage{polyglossia}
\setmainlanguage{russian}
\setotherlanguages{english}

\setmainfont[Mapping=tex-text]{Times New Roman}
\setmainfont[Mapping=tex-text]{Times New Roman}
\newfontfamily{\cyrillicfonttt}{Courier New}
\setmonofont{Courier New}

%нумерация справа и колонтитулы справа вверху
\usepackage{fancyhdr}
\usepackage[left=25mm,right=10mm,top=20mm,bottom=20mm,bindingoffset=0cm]{geometry}%

\usepackage{amsfonts}
\usepackage{amssymb}
\usepackage{amsmath}
\usepackage{amsthm}

\usepackage{calc}
\usepackage{ifthen}
\usepackage{graphicx}
\usepackage{array}
\usepackage{pdfpages}
\usepackage{longtable}
\usepackage{indentfirst}
\usepackage[unicode=true]{hyperref}
\usepackage{listings}
\usepackage{color}

\definecolor{mygreen}{rgb}{0,0.6,0}
\definecolor{mygray}{rgb}{0.5,0.5,0.5}
\definecolor{mymauve}{rgb}{0.58,0,0.82}

\lstset{ %
  backgroundcolor=\color{white},   % choose the background color; you must add \usepackage{color} or \usepackage{xcolor}; should come as last argument
  basicstyle=\linespread{0.8}\footnotesize\ttfamily,        % the size of the fonts that are used for the code
  breakatwhitespace=false,         % sets if automatic breaks should only happen at whitespace
  breaklines=true,                 % sets automatic line breaking
  captionpos=b,                    % sets the caption-position to bottom
  commentstyle=\color{mygreen},    % comment style
  %deletekeywords={...},            % if you want to delete keywords from the given language
  escapeinside={\%*}{*)},          % if you want to add LaTeX within your code
  extendedchars=true,              % lets you use non-ASCII characters; for 8-bits encodings only, does not work with UTF-8
  frame=single,	                   % adds a frame around the code
  keepspaces=true,                 % keeps spaces in text, useful for keeping indentation of code (possibly needs columns=flexible)
  keywordstyle=\color{blue},       % keyword style
  language=Scala,                 % the language of the code
  morekeywords={*,...},           % if you want to add more keywords to the set
  numbers=left,                    % where to put the line-numbers; possible values are (none, left, right)
  numbersep=5pt,                   % how far the line-numbers are from the code
  numberstyle=\tiny\color{mygray}, % the style that is used for the line-numbers
  rulecolor=\color{black},         % if not set, the frame-color may be changed on line-breaks within not-black text (e.g. comments (green here))
  showspaces=false,                % show spaces everywhere adding particular underscores; it overrides 'showstringspaces'
  showstringspaces=false,          % underline spaces within strings only
  showtabs=false,                  % show tabs within strings adding particular underscores
  stepnumber=2,                    % the step between two line-numbers. If it's 1, each line will be numbered
  stringstyle=\color{mymauve},     % string literal style
  tabsize=2,	                   % sets default tabsize to 2 spaces
  title=\lstname                   % show the filename of files included with \lstinputlisting; also try caption instead of title
}

\usepackage[singlelinecheck=false,labelsep=endash]{caption}
\captionsetup[table]{justification=justified}
\captionsetup[figure]{justification=centering}

\usepackage{titlesec}
\titleformat{\chapter}[block]{\centering\normalfont\LARGE\bfseries}{\thechapter.}{1ex}{}{}
\titlespacing{\chapter}{0pt}{0em}{2em}

\usepackage[title, titletoc]{appendix}
%\renewcommand{\appendixname}{Приложение}% Change "chapter name" for Appendix chapters
%\renewcommand{\cftchapdotsep}{\cftdotsep}

\usepackage{mathpartir}

\makeatletter
\let\ps@plain\ps@fancy              % Подчиняем первые страницы каждой главы общим правилам
\makeatother
\pagestyle{fancy}
\fancyhf{}
\fancyfoot[C]{\thepage}
\renewcommand{\headrulewidth}{0pt}
\renewcommand{\footrulewidth}{0pt}
\renewcommand{\baselinestretch}{1.5}
\newcommand{\headertext}[1]{\fancyhead[R]{\tiny{#1}}}

%% Список литературы
\makeatletter
\bibliographystyle{utf8gost71s}     % Оформляем список литературы по ГОСТ 7.1
                                    % (ГОСТ Р 7.0.11-2011, 5.6.7)
\renewcommand{\@biblabel}[1]{#1.}   % Заменяем список литературы с квадратных
                                    % скобок на точку
\makeatother

%\frenchspacing %% изменение расстояние до и после точек в ряде случаев

\renewcommand{\theenumi}{\arabic{enumi}}
\renewcommand{\theenumii}{\arabic{enumii}}
\renewcommand{\theenumiii}{\arabic{enumiii}}
\renewcommand{\theenumiv}{\arabic{enumiv}}

\renewcommand{\labelenumi}{\theenumi.}
\renewcommand{\labelenumii}{\theenumi.\theenumii.}
\renewcommand{\labelenumiii}{\theenumi.\theenumii.\theenumiii.}
\renewcommand{\labelenumiv}{\theenumi.\theenumii.\theenumiii.\theenumiv.}


%\newenvironment{annotation}{\textbf{Аннотация.} \textit}{}
\theoremstyle{plain}
\newtheorem*{annotation}{Аннотация}