\chapter{Анализ проблематики генерации документов на основе шаблонов}
\label{chapter1}


В данном разделе приводятся теоретические аспекты проблемной области решаемой задачи. Приведен сравнительный анализ формальных средств описания шаблонов объектов и изучены правила подстановки при формировании конечных файлов. Также описаны технологии генерации Word- и PDF-документов. В подразделе «Выводы» подведены итоги анализа предметной области. В конце раздела формулируется цель научно-исследовательской работы, а также перечисляются задачи, которые необходимо выполнить для достижения поставленной цели.






\section{1.1. Изучение и сравнительный анализ формальных средств описания шаблонов объектов. Изучение правил подстановки при формировании конечных файлов}


В данном разделе приводятся теоретические аспекты проблемной области решаемой задачи. Приводится сравнительный анализ средств описания шаблонов объектов. Также приводятся изученные правила подстановки при формировании конечных файлов.



Исчисление шаблонов является результатом глубокого пересмотра 50-летней разработки. Основная цель ~--- обеспечить унифицирующий подход, связывающий различные стили программирования и парадигмы при помощи сопоставления шаблонов. Полная оценка эффективности данного представления требует глубокого знания
многочисленных теоретических областей информатики, такие как лямбда-исчисление, терминологическая перепись и теория типов.

Исчисление на основе шаблонов является новым основанием для вычислений, в которых выражения мощностей функций и структур данных плодотворно сочетаются в рамках функций сопоставления шаблонов. Лучшие из существующих основ сосредоточены на любых функциях (в $ \lambda $ - исчислении ) или на структурах данных (в машинах Тьюринга), или же на обоих перечисленных (как в объектной ориентации). 

Чтобы найти баланс между функциями и структурами, начнем с теории функций, λ - исчисления. Короче говоря, каждый член чистого $ \lambda $ -исчисления является либо переменным, либо применением, либо $ \lambda $ -абстракцией. Оценка дается одним правилом, которое заменяет аргумент функции связанной переменной. Чистое $ \lambda $ -исчисление способно кодировать структуры данных как абстракции. Но такое кодирование не является однородным, поскольку нет средств для выделения атомов из соединений.
Таким образом, первым шагом является добавление некоторых конструкторов, которые являются атомами. Из таких атомов могут быть созданы структуры данных, посредством применения. Затем добавляются некоторые операции, которые действуют на структуры данных, например, для сравнения конструкторов или восстановления компонентов. Результирующего исчисления должно быть достаточно для определения основных интересующих запросов. 

Альтернативным средством воздействия на структуры данных является использование сопоставления шаблонов.
Шаблоны могут использоваться для описания внутренней структуры данных, то есть ее формы, и для обозначения ее частей, которые затем доступны для использования.

Концептуальный шаблон имеет набор входных и выходных параметров, а также
набор начальных значений. После инициализации этих параметров получается частная
модель некоторой предметной области. Таким образом, технология концептуальных
шаблонов позволяет экспертам создавать модели сложных систем, не углубляясь в язык
системной динамики.

Формальное понятие шаблона на языке теории множеств представлено в виде
следующего множества формальных шаблонов $ P= \lbrace E,Sign2,R,Fn\rbrace $. $ E=\lbrace L,A,C\rbrace $
~--- множество элементов шаблонов, где $ L $ ~--- множество уровней шаблонов (наличия уровня подразумевает наличие входящих и исходящих потоков); A ~--- множество  переменных шаблонов; С ~--- множество  констант 	шаблонов. $ Sign2 $ ~--- множество 	вторичных признаков распознавания (термины, описывающие шаблон). $ R $ ~--- множество  отношений внутри шаблонов. $ Fn $ ~--- множество  законов функционирования переменных в шаблонах.

Поскольку шаблоны могут иметь достаточно сложную внутреннюю структуру,
то на разработку модели сложной системы, без технологии концептуальных шаблонов,
экспертам может понадобиться достаточно большой промежуток времени. Кроме того,
если эксперты не обладают достаточными знаниями о языке системной динамики и
опытом построения системно-динамических моделей, то понадобится привлечение
группы экспертов в области системно-динамического моделирования. Как следствие,
возникает проблема взаимопонимания экспертов, а также увеличение финансовых затрат. Использование технологии концептуальных шаблонов позволяет решить выше
указанные проблемы.\cite{henk_lambda}

Далее рассматриваются правила подстановки в $ \lambda $ ~--- исчислении.

Процесс вычисления термов заключается в подстановке аргументов во все функции. Выражения вида:

$$ (\lambda x.M) N $$

Заменяются на

$$ M[x=N] $$

Эта запись означает, что в терме $ M $ все вхождения $ x $ заменяются на терм $ N $. Этот процесс называется редукцией терма. А выражения вида $ (\lambda	x.M)N $ называются редексами.

При подстановке необходимо следить за тем, чтобы у нас не появлялись лишние связывания переменных. Например рассмотрим такой редекс:

$$ (\lambda x y.x) y $$

После подстановки за счёт совпадения имён переменных мы получим тождественную функцию:

$$ \lambda y.y $$

Переменная $ y $ была свободной, но после подстановки стала связанной. Необходимо исключить такие случаи. Поскольку с ними получается, что имена связанных переменных в определении функции влияют на её смысл. Например смысл такого выражения

$$ (\lambda xz.x) y $$

После подстановки будет совсем другим. Но мы всего лишь изменили обозначение локальной переменной $ y $ на $ z $. И смысл изменился, для того чтобы исключить такие случаи пользуются переименованием переменных или $ \alpha $ -преобразованием. Для корректной работы функций необходимо следить за тем, чтобы все переменные, которые были свободными в аргументе, остались свободными и после подстановки.

Процесс подстановки аргументов в функции называется $ \beta $ -редукцией. В редексе $ (\lambda x.M)N $ вместо свободных вхождений $ x $ в $ M $ мы подставляем $ N $. Посмотрим на правила подстановки:

$$ x[x=N] & \Rightarrow N $$

$$ y[x=N] & \Rightarrow y $$

$$ (PQ)[x=N] & \Rightarrow (P[x=N] Q[x=N])$$

$$ (\lambda y.P)[x=N] & \Rightarrow (\lambda y.P[x=N]), y \notin FV(N)$$

$$ (\lambda x.P)[x=N]  & \Rightarrow (\lambda x.P) $$

Первые два правила определяют подстановку вместо переменных. Если переменная совпадает с той, на место которой мы подставляем терм $ N $ , то мы возвращаем терм $ N $, иначе мы возвращаем переменную:

$$ x[x=N] \Rightarrow N $$

$$ y[x=N] \Rightarrow y $$

Подстановка применения термов равна применению термов, в которых произведена подстановка:

$$ (PQ)[x=N] \Rightarrow (P[x=N] Q[x=N]) $$

При подстановке в лямбда-функции необходимо учитывать связность переменных. Если переменная аргумента отличается от той переменной на место которой происходит подстановка, то мы заменяем в теле функции все вхождения этой переменной на $ N $:

$$ (\lambda y.P)[x=N] \Rightarrow (\lambda y. P[x=N]), y \notin FV(N) $$

Условие $ y \notin FV(N) $ означает, что необходимо следить за тем, чтобы в $ N $ не оказалось свободной переменной с именем $ y $, иначе после подстановки она окажется связанной. Если такая переменная в $ N $ всё-таки окажется мы проведём $ \alpha $ -преобразование в терме $ \lambda y. M $ и заменим $ y $ на какую-нибудь другую переменную.

В последнем правиле мы ничего не меняем, поскольку переменная $ x $ оказывается связанной. А мы проводим подстановку только вместо свободных переменных:

$$ (\lambda x.P)[x=N] \Rightarrow (\lambda x.P) $$ \cite{barry_jay}

\section{1.2. Изучение технологии компиляции PDF-файлов из исходных текстов на LaTeX}


В данном разделе описывается процесс компиляции PDF-файлов из исходных текстов на LaTeX. Также приводится структура и особенности вёрстки документов на LaTex.


LaTeX-документ (расширение tex) — это обычный текстовый файл, в котором содержится и некоторый объём команд для LaTeX процессора. В каком-то смысле это программа, по выполнении которой получается качественно оформленная печатная или электронная копия документа. \cite{ibm_latex}

Типичный LaTeX-документ имеет следующую структуру:

Листинг 1. Структура LaTeX-документа
\begin{verbatim}
\documentclass{article} %Определение класса документа
    %Заголовок документа
\begin{document}
    %Содержание документа
\end{document}
\end{verbatim}

\textbf{Классы LaTeX-документов}

Класс -  это некоторый базовый набор команд определяющий внешний вид будущего документа. Файлы классов в LaTeX имеют расширение .sty . В дистрибутивах доступны некоторые стандартные классы в частности article, report и некоторые другие. Практически все классы принимают аргументы, например команда:

\begin{verbatim}
\documentclass[11pt,a4paper,oneside]{report}
\end{verbatim}


создаст документ класса report, с форматом бумаги A4, базовым размером шрифта в 11pt и полями для односторонней печати.
Как правило пользователь использует один из стандартных стилей и модифицирует внешний вид документа командами, которые он добавляет в заголовке документа, однако если таких команд много, то можно создать собственный стилевой файл.\cite{lvovsky_2003}
 
Листинг 2. Исходный файл HelloWorld.tex
\begin{verbatim}
\documentclass[12pt,a4paper]{scrartcl}
\usepackage[utf8]{inputenc}
\usepackage[english,russian]{babel}
\usepackage{indentfirst}
\usepackage{misccorr}
\usepackage{graphicx}
\usepackage{amsmath}
\begin{document}
Здравствуй, Мир!!!
\end{document}
\end{verbatim}

На первой строке загружается класс документа scrartcl. Этот класс входит в набор KOMA-Script — современный пакет с отличной документацией и богатыми возможностями. На следующих строках загружаются стилевые файлы, необходимые для "русификации" документа:
	
\begin{itemize}
\item inputenc — для выбора кодировки текста;
\item babel — пакет для локализации;
\item indentfirst — красная строка для первого параграфа;
\item misccorr — пакет с дополнительными настройками для соответствия правилам отечественной полиграфии.
\end{itemize}

Стили graphicx и amsmath отвечают за вставку картинок и отображение математической нотации.

Сам текст документа набирается внутри окружения document, которое начинается с команды \verb|\begin{document}| и заканчивается конструкцией \verb|\end{document}|. Параграфы в тексте разделяются друг от друга пустой строкой. После создания файла HelloWorld.tex, его можно скомпилировать с помощью программы pdflatex и посмотреть полученный в результате PDF-файл HelloWorld.pdf, как показано ниже

\begin{verbatim}
pdflatex HelloWorld.tex
okular HelloWorld.pdf
\end{verbatim}

\textbf{Создание титульного листа}

Перед началом работы следует попытаться найти готовый класс LaTeX, который учитывает все правила к оформлению научных публикаций, установленные в ВУЗе. Если такой файл найдётся (у других студентов или в администрации ВУЗа), то задачу по вёрстке документа можно считать решённой, что позволит сразу перейти к набору текста.
К сожалению, для российских ВУЗов такие файлы встречаются крайне редко, поэтому мы выполним оформление титульного листа "вручную", вставляя в него пробелы, выверенные линейкой на твёрдой копии образца.

После оформления титульного листа можно переходить к набору текста. Пакет LaTeX берёт на себя работу по оформлению заголовков разделов и их автоматической нумерации. Достаточно только указать, что в данном месте начался новый раздел с помощью команд \verb|\section| (раздел), \verb|\subsection| (подраздел) и \verb|\subsubsection| (подподраздел).

Весь текст, который находится за символом  \% , считается комментарием, и поэтому не выводится при печати. Символ процента можно вывести с помощью команды \verb| \% |, а символ  \~  формирует неразрывный пробел.
Кроме символа процента необходимо экранировать символы \{ \} \$ \& \# \_. Также специальным является и символ \verb|\|.

Окружение enumerate формирует нумерованное перечисление. Аналогично ненумерованное перечисление создаётся с помощью окружения \verb|itemize|.
Обратите внимание на метки, поставленные с помощью команды \verb|\label| вслед за заголовками. Используя эти метки, можно с помощью команд \verb|\ref| и \verb|\pageref| сослаться на номер и страницу соответствующего раздела. Для выставления правильной нумерации в ссылках компилятору потребуется выполнить два прохода:

\begin{verbatim}
pdflatex Kurs.tex
pdflatex Kurs.tex
\end{verbatim}

\textbf{Математика в LaTeX}

Как неоднократно говорилось в статьях цикла "Каталог классов и стилей LaTeX" этот издательский пакет был изначально оптимизирован для набора математических публикаций. Но перед тем как начать набирать математические выражения, нам предстоит изучить соответствующую TeX-нотацию.

Листинг 3. Пример математической нотации

Решение квадратного уравнения 

\begin{verbatim}
\(ax^2+bx+c=0\):
\begin{equation}\label{eq:solv}
 x_{1,2}=\frac{-b\pm\sqrt{b^2-4ac}}{2a}
\end{equation}

\end{verbatim}

Можно сослаться на уравнение \verb|~\eqref{eq:solv}|.

\textbf{Вставка кода}

Для добавления неформатируемых фрагментов текста (например, программного кода) в LaTeX-лучше всего использовать окружение \verb|verbatim|, как показано в листинге 4.

Листинг 4. Пример включения неформатируемого текста в LaTeX-документ

\begin{verbatim}
\begin{verbatim}
for alpha:=-90 step 3 until 0:
  label(btex IBM developerWorks etex
    scaled (5*(1+alpha/100)) rotated alpha,(0,0))
  withcolor (max(1+alpha/45,0)*red+
    min(-alpha/45,2+alpha/45)*green+
    max(-alpha/45-1,0)*blue);
endfor;
\end{verbatim}
\verb|\end{verbatim} |

\textbf{Библиография}

В конце любой научной работы обязательно должна присутствовать библиография, которую проще всего создать с помощью окружения thebibliography, как показано в листинге 5.

Листинг 5. Создание библиографии

\begin{verbatim}
\begin{thebibliography}{9}
\bibitem{Knuth-2003}Кнут Д.Э. Всё про \TeX. \newblock --- Москва:
Изд. Вильямс, 2003. 550~с.
\bibitem{Baldin-2008}Балдин Е.М. Компьютерная типография
\LaTeX. \newblock --- Санкт-Петербург: Изд. БХВ-Петербург,
2008. 302~с.
\end{thebibliography}
\end{verbatim}

Команды \verb| \bibitem | формируют библиографические ссылки, на которые можно ссылаться с помощью команды \verb| \cite |, как показано ниже (ссылаться можно даже из тех фрагментов текста, которые располагаются выше определения ссылки):

Для изучения «внутренностей» \verb|\TeX{} | необходимо изучить \verb| ~\cite{Knuth-2003} |, а для использования \verb| \LaTeX{} | лучше почитать\verb| ~\cite{Baldin-2008}.|
Как и в случае с перекрёстными ссылками для правильного отображения библиографических ссылок, исходный документ необходимо скомпилировать дважды.
В заключении можно сказать, что сверстать документ так, чтобы его было приятно и удобно читать – это далеко не простая задача. Пакет LaTeX позволяет получить приемлемый результат за разумный промежуток времени без необходимости привлечения специалиста-верстальщика. Однако создание сложных текстов требует временных затрат на изучение возможностей LaTeX. 




\section{1.3. Изучение средств автоматизации MS Word. Изучение возможностей генерации Word-документов программным путем}


В данном разделе описываются основные средства и возможности генерации документов Word при помощи программных средств. Приводятся результаты автоматического создания документов по заданному шаблону  на основе массива имеющихся данных.


\subsection{1.3.1. Генерация документов Word по шаблону с использованием технологии COM Interoperability}


В данном разделе описывается процесс генерации документов Word по заданному шаблону с использованием технологии COM Interoperability, позволяющей получить доступ к объектам Word.


Существует множество способов вывести данные в документ Word. Количество подходов для решения данной проблемы порождает необъятную тему возможностей как самого Word, так и технологий, способных взаимодействовать с ним. 

    Если свести задачу к простейшему варианту - работе с простыми строками (типовая задача в крупных предприятиях - вставка дат, цифр, фио и тому подобных вещей) - самым простым решением на языке C# будет являться механизм COM  Interoperability( сокращенно Interop). Для этого потребуется открыть из кода C# шаблон Word и что-то в него вставить. Далее необходимо приготовить шаблон .dot, используя тот же самый Word, и обращаться к нему, используя Interop. То есть необходимо запускать отдельный exe-процесс самого Word и через специальный интерфейс управлять им. Такой интерфейс находится в списке подключаемых библиотеках, поставляемых вместе с Office. Ниже перечислены следующие этапы, необходимые для реализации описанного процесса:

\begin{enumerate}
\item Подключить необходимые библиотеки (C#)
\item Открыть шаблон Word
\item Определить необходимое место для подстановки
\item Вставить в необходимое место строку с информацией

\end{enumerate}

Далее будет немного подробней описан процесс реализации. В начале в код добавляются необходимые библиотеки:

\begin{verbatim}
using Word = Microsoft.Office.Interop.Word;
using System.Reflection;
\end{verbatim}

Далее, по вышеописанной инструкции, идёт работа с самим интерфейсом Word. При этом стоит учитывать следующие принципиально важные моменты:

Самый простой и примитивный вариант работы с шаблоном - поиск и замена строки в документе Word. Весь процесс заключается в том, чтобы добавить текстовую метку вроде \verb|@@nowDate| и заменить её на необходимое значение.

Далее основная часть работы ведется с объектом Range. Данный объект представляет некую смежную область в документе, определяется начальным и конечным символами и может включать в себя все что угодно - от пары символов, до таблиц, закладок и т.д. Подобно тому, как закладки используются в документе, объекты Range используются в процедурах Visual Basic для идентификации определенных частей документа. Однако, в отличие от закладки, объект Range существует только при запущенной процедуре. Объекты диапазона не зависят от выбора. Соответственно есть возможность определять и манипулировать диапазоном без изменения выбора. Вы также можете определить несколько диапазонов в документе, в то время как для каждой панели может быть только один выбор. 

Соответственно необходимо получить Range для всего документа, найти нужную строку внутри него, получить Range для этой строки и уже внутри этого последнего диапазона заменить текст на требуемый.

Если строку надо просто заменить, то сойдет простейший способ:

\begin{verbatim}
_range.Text = “Это текст заменит содержимое Range”
\end{verbatim}

Но так как Range является универсальный контейнером для любого куска документа Word, то его возможности неизмеримо шире и могут быть рассмотрены на более высоких уровнях.

Если же необходимо просто вставить в начало документа:

\begin{verbatim}
object start = 0;
object end = 0
_currentRange = _document.Range(ref start, ref end);
\end{verbatim}

Таким образом работает один из подходов генерации документов Word по шаблону, используя доступ к объектам Word.



\subsection{1.3.2. Генерация документов Word по шаблону с использованием библиотеки C\#}


В данном разделе описывается возможность генерации документов Word по заданному шаблону при помощи подключаемой библиотеки на языке C\#


EasyDox.dll – это бесплатная .NET библиотека для генерации вордовских документов (docx) по шаблону. Использовать ее очень просто:

\begin{verbatim}
var fieldValues = new Dictionary <string, string> {
    {"№ договора", "123-456/АГ"},
    {"Сторона 1",  "ООО «Ромашка»"},
    {"Сторона 2",  "ЗАО «Тюльпан»"},
    {"Подписант 1", "Иванов И.П."},
    {"Должность 1", "генеральный директор"},
    {"Основание 1", "Устав"},
};
var engine = new Engine ();

engine.Merge ("c:\\template.docx", fieldValues,"c:\\output.docx");
\end{verbatim}

Функция Merge читает указанный файл шаблона и подставляет в него значения полей, заданных параметром \verb|fieldValues|, и затем сохраняет результат в \verb|"c:\output.docx"|.

Ниже представлен список преобразований:

\begin{center}
\begin{tabular}{ccc}
\textbf{Преобразование} & \textbf{Что делает} \\
(родительный) & Ставит предшествующую позицию в родительный падеж. \\
(цифрами и прописью) & Преобразует число в запись суммы в рублях цифрами и прописью. \\

\end{tabular}
\end{center}

Текущая (первая) версия библиотеки включает в себя минимум преобразований. В последующих версиях будут добавлены остальные падежи для русского и украинского, а также возможность склонения по родам.

Библиотека расширяема и позволяет добавлять собственные преобразования. Набор пользовательских преобразований (функций) передается в конструктор класса Engine.
Поля в колонтитулах (верхних и нижних) пока не обрабатываются.

Библиотека собрана под платформу AnyCPU (MSIL). Не требует установки OpenXml SDK. Все классы потокобезопасны.


\section{1.4. Изучение средств автоматизации MS Word. Изучение возможностей генерации Word-документов программным путем}


В данном разделе  описываются основные функции и  преимущества выбранной программной технологии для разработки веб-приложений на платформе .NET. Также  описываются функциональные возможности ASP.NET Core MVC, упрощающие создание веб-API и веб-приложений.


ASP.NET Core является кроссплатформенной, высокопроизводительной средой с открытым исходным кодом для создания современных облачных приложений, подключенных к Интернету. ASP.NET Core позволяет выполнять следующие задачи:

\begin{itemize}
\item Создавать веб-приложения и службы, приложения IoT и серверные части для мобильных приложений.
\item Использовать избранные средства разработки в Windows, macOS и Linux.
\item Выполнять развертывания в облаке и локальной среде.
\item Работать в .NET Core или .NET Framework.

\end{itemize}

\subsection{1.4.1. Преимущества использования ASP.NET}


В данном разделе описываются основные преимущества использования ASP.NET по сравнению с известными аналогами. Обосновывается выбор данного программного средства для реализации поставленной задачи.


Миллионы разработчиков использовали ASP.NET (и продолжают использовать) для создания веб-приложений. ASP.NET Core является модификацией ASP.NET с внесенными архитектурными изменениями, формирующими более рациональную и модульную платформу.

ASP.NET Core предоставляет следующие преимущества:

\begin{itemize}
\item Единое решение для создания пользовательского веб-интерфейса и веб-API.
\item Интеграция современных клиентских платформ и рабочих процессов разработки.
\item Облачная система конфигурации на основе среды.
\item Встроенное введение зависимостей.
\item Упрощенный, высокопроизводительный и модульный конвейер HTTP-запросов.
\item Возможность размещения в IIS или в собственном процессе.
\item Возможность запуска на платформе .NET Core, которая поддерживает параллельное управление версиями приложения.
\item Инструментарий, упрощающий процесс современной веб-разработки.
\item Возможность сборки и запуска в ОС Windows, macOS и Linux.
\item Открытый исходный код и ориентация на сообщество.

\end{itemize}

\cite{asp.net_core}

ASP.NET Core поставляется полностью в виде пакетов NuGet. Это позволяет оптимизировать приложения для включения только необходимых пакетов NuGet. За счет небольшого размера контактной зоны приложения доступны такие преимущества, как более высокий уровень безопасности, минимальное обслуживание и улучшенная производительность.

\textbf{WCF и ASP.NET Web API}

WCF является единой моделью программирования (Майкрософт) для построения ориентированных на службы приложений.Она позволяет разработчикам построить безопасные надежные решения с поддержкой транзакций и возможностью межплатформенной интеграции и взаимодействия с существующими инвестициями.

ASP.NET Web API — платформа .NET Framework, которая позволяет легко создавать службы HTTP, доступные для широкого круга клиентов, включая браузеры и мобильные устройства.ASP.NET Web API — это идеальная платформа для сборки REST-приложений на базе .NET Framework. Далее рассматривается сравнительная характеристика обоих технологий с целью решить, какая из них лучше подходит под конкретные требования.

В следующей таблице описаны основные возможности каждой из технологий:

\begin{center}
\begin{tabular}{| p{200pt} | p{200pt} |}
\hline
\textbf{WCF} & \textbf{ASP.NET Web API} \\ \hline
Включает службы сборки, которые поддерживают несколько транспортных протоколов (HTTP, TCP, UDP и пользовательский транспорт) и допускают переключение между ними. & Только HTTP. Идеальная модель программирования для HTTP. Наиболее подходит для доступа из различных браузеров, мобильных устройств и т. д., обеспечивая более широкий охват. \\ \hline
Включает службы сборки, которые поддерживают разные кодирования (текст, MTOM и двоичные) одного типа сообщений и допускают переключение между ними & Позволяет создавать сетевые API-интерфейсы, которые поддерживают большое количество различных типов содержимого, в том числе XML, JSON и т. д. \\ \hline
Поддерживает создание служб по таким стандартам WS-*, как надежный обмен сообщениями, транзакции и безопасность сообщений. & Использует основные протоколы и форматы, такие как HTTP, WebSockets, SSL, JQuery, JSON и XML.Отсутствует поддержка протоколов высокого уровня, таких как надежный обмен сообщениями и транзакции. \\ \hline
Поддерживает шаблоны обмена сообщениями «запрос-ответ», «односторонний» и «дуплексный». & HTTP работает через «запрос-ответ», однако поддерживаются дополнительные шаблоны через интеграцию SignalR и WebSockets. \\ \hline

Службы WCF SOAP могут быть описаны в языке WSDL, что позволяет автоматическим средствам создавать прокси клиентов даже для служб со сложными схемами. & Имеются различные способы описания Web API — от автоматически формируемых html-страниц справки с описанием фрагментов до структурированных метаданных для интеграции API в OData. \\ \hline
Поставляется вместе с платформой .NET Framework. & Поставляется вместе с платформой .NET Framework, но имеет открытый код и доступна также по внешним каналам как независимая загрузка.\\ \hline


\end{tabular}
\end{center}

Исходя из этого можно сделать выводы о выборе подходящей технологии. Как видно, WCF лучше использовать для создания безопасных и надежных веб-служб. При этом ASP.NET Web API лучше использовать для создания служб на основе HTTP, доступных для множества клиентов. ASP.NET Web API следует использовать при создании и разработке новых служб в стиле REST. Хотя WCF предоставляет некоторую поддержку написания служб в стиле REST, поддержка REST в ASP.NET Web API более полная и все последующие улучшения функций REST будут вноситься в ASP.NET Web API.
\cite{rihter} \cite{heilsberg}
\subsection{1.4.2. Создание веб-API и пользовательского веб-интерфейса с помощью ASP.NET Core MVC}


В данном разделе описываются функциональные возможности ASP .NET Core MVC, играющие важную роль в выборе средств для создания веб-API и различного рода веб-приложений.


ASP.NET Core MVC предоставляет функциональные возможности, упрощающие создание веб-API и веб-приложений:

\begin{itemize}

\item Шаблон Model-View-Controller (MVC) помогает сделать веб-API и веб-приложения тестируемыми.
\item Страницы Razor (новинка в версии 2.0) — это основанная на страницах модель программирования, которая упрощает и повышает эффективность создания пользовательского веб-интерфейса.
\item Синтаксис Razor предоставляет эффективный язык для страниц Razor и представлений MVC.
\item Вспомогательные функции тегов позволяют серверному коду участвовать в создании и отображении HTML-элементов в файлах Razor.
\item Благодаря встроенной поддержке нескольких форматов данных и согласованию содержимого веб-API становятся доступными для множества клиентов, включая браузеры и мобильные устройства.
\item Привязка модели автоматически сопоставляет данные из HTTP-запросы с параметрами метода действия.
\item Проверка модели автоматически выполняет проверку на стороне сервера и клиента.
\item Помимо прочего, ASP.NET Core легко интегрируется с широким спектром клиентских платформ, включая AngularJS, KnockoutJS и Bootstrap. \cite{freeman}
\end{itemize}

\section{1.5. Выводы}


В данном разделе приводятся заключения относительно анализа предметной области и существующих методов решения поставленной задачи. Сформулированы основные выводы насчёт практического подхода к решению поставленной задачи.



В ходе аналитической части учебно-исследовательской работы были сделаны следующие заключения:

\begin{enumerate}
\item Для генерации PDF файлов будет применяться стандартная технология компиляции из исходных текстов на LaTeX с использованием дополнительно подключаемых пакетов.
\item Для генерации Word-документов на основе шаблонов будет использована технология Interop, позволяющая работать непосредственно с объектами Word.
\item Для разработки веб-сервиса была выбрана технология ASP.NET Web API 2.0.
\end{enumerate}

\section{1.6. Формальная постановка задачи}

Целями курсового проекта являются:

\begin{enumerate}
\item Реализация механизмов генерации Word-файлов.
\item Реализация механизмов генерации PDF-файлов.
\item Реализация веб-сервиса, работающего с разработанными механизмами генерации документов, создание простейшего пользовательского интерфейса.
\end{enumerate}

    Для достижения первой цели необходимо изучить литературу, связанную с формальными средствами описания шаблонов объектов и правилами подстановки. Необходимо изучить и провести сравнительный анализ технологий генерации Word файлов программным путём. В конце необходимо провести тестирование и оценку производительности разработанного механизма.
    
    Для достижения второй цели необходимо изучить литературу, описывающую процесс компиляции PDF файлов из исходных текстов на LaTeX. Также необходимо автоматизировать данный процесс. В конце необходимо провести тестирование и оценку производительности разработанного механизма.
    
    Для достижения третьей цели необходимо изучить литературу и провести сравнительный анализ серверных технологий для разработки веб-приложений на платформе .NET. Необходимо спроектировать систему программных веб-интерфейсов, разработать функциональную структуру информационной системы, а также провести тестирование с помощью выбранных утилит.
